%%%%  Abstract
\begin{abstract}

Asymmetric Dark Matter models are based on the observation that the fractional densities of baryonic matter and dark matter are so close. An asymmetric
dark matter model hypothesises that in the dark matter sector there is an asymmetry in some global "dark baryon number" that is then related in some
way to the asymmetry in the visible sector baryons. Many models have been proposed that both generate the dark sector asymmetery and transfer the asymmetry
to the visible sector using more than a few different mechanisms. This paper focuses on the Darkogenesis model in which the dark asymmetry is generated
througha first-order phase transition using dark sphalerons. Once generated the asymmetry can be transfered by either of a low scale mechanism where the
dark asymmetry is transfered though baryon number violating higher order operators, and a high scale mechanism in which electroweak sphalerons are responsible 
for the transfer. This model is able to generate asymmetries that corespond well to observed relic densities in both sectors and do not have large effects on 
well measure standard model observables. 

\end{abstract}
