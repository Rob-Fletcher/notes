%%%%%  Conclusion
\section{Conclusions}

We have described the motovation for pursuing models that relate observed relic densities of baryonic and dark matter. Through transfer mechanisms 
these models are able to reprocess an asymmetry in one sector to the other. In the case of the Darkogenesis model, the initial asymmetry was generated
in a first-order phase transition through electroweak sphalerons. This asymmetry could then be communicated to the visible sector by one of two ways: 
\begin{itemize}
  \item If the dark phase transition is after the electroweak one the asymmetry must be processed directly into visible baryons by higher order baryon 
        number violating operators. 
  \item If the dark phase transition is before the electroweak one, it is possible to add a messenger sector that carries both the dark baryon number
        and the electroweak $SU(2)_L$. These messengers can then transfer the asymmetry via electroweak sphalerons.
\end{itemize}

This model demonstrates that there can be enough asymmetry generation and transfer to account for the observed relic densities of the visible and dark sectors.
It also predicts a dark matter mass of $\sim 1-15 \text{ }GeV$ and its interference with precision measurements of the standard model electroweak theory are within
observable bounds. This is a very interesting model as it is able to explain some of the big modern problems in cosmology and is easily embedded in a supersymmetric
theory such as (n)MSSM.
