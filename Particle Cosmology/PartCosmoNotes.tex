
\documentclass{article}
\usepackage{amssymb, amsmath}
\numberwithin{equation}{section}
\title{Notes on Particle Cosmology}
\author{Rob Roy Fletcher, Professor Mark Trodden}

%%%%%%%%----  Macros ----%%%%%%%%%%%%%%%
\newcommand{\Lagr}{\mathcal{L}}  %lagrangian
\newcommand{\dmu}{\partial_{\mu}} 
\newcommand{\mnk}{\eta_{\mu \nu}}  %minkowski metric lower indices
\newcommand{\mtr}{g_{\mu \nu}}     % metric lower indices
\newcommand{\Mnk}{\eta^{\mu \nu}}
\newcommand{\Mtr}{g^{\mu \nu}}
\newcommand{\beq}{\begin{equation}}
\newcommand{\eeq}{\end{equation}}
  
%%%%%%%---- Begin Document ----%%%%%%%%%%
\begin{document}
\maketitle
\section{August 29, 2013}
\subsection{Review of General Relativity}

Special Relativity can be thought of as transformations that leave invariant the interval,      $(c=1)$
\beq
    ds^2 = -dt^2 + dx^2 + dy^2 + dz^2
    = \mnk dx^\mu dx^\nu
\eeq

What transformations leave $ds^2$ invariant?

$\longrightarrow$   The Lorentz Transformations.
\beq
    x^{\mu'} = \Lambda^{\mu'}_{\mu} x^\mu
    \Lambda^{\mu'}_{\mu} = \frac{\partial x^{\mu'}}{\partial x^{\mu}}
\eeq

To ensure our physical laws are Lorentz invariant we write them in terms of vectors ({\it contravariant vectors}) and covectors ({\it covariant vectors}).

A vector transforms as $V^{\mu'} = {\Lambda^{\mu'}}_{\mu} V^\mu$.

A covector transforms as $W_{\mu'} = {\Lambda^{\mu}}_{\mu'} W_\mu$.

A general $(m,n)$ Tensor ${T^{\mu_{1},\mu_{2}...\mu_{m}}}_{\nu_{1},\nu_{2}...\nu_{n}}$ transforms as
${T^{\mu_{1}',\mu_{2}'...\mu_{m}'}}_{\nu_{1}',\nu_{2}'...\nu_{n}'} =\Lambda \Lambda ... \Lambda ... \Lambda {T^{\mu_{1},\mu_{2}...\mu_{m}}}_{\nu_{1},\nu_{2}...\nu_{n}}$

e.g. Electromagnetism:

\begin{centering}
$A^\mu = (\phi, \vec{A})$\ \ \ \ \ \ $J^\mu = (q, \vec{J})$

$F\equiv \dmu A_\nu - \partial_{\nu} A_\mu$

$\dmu F^{\nu \mu} = J^\nu$  \ \ \ \ \ \ $\partial_{[\mu} F_{\rho \nu ]} = 0$ 

\end{centering}



How to write down physical laws in this space? ($ds^2 = -dt^2 + dx^2 + dy^2 + dz^2$)
We want to leave $ds^2$ invariant under any transformation of coordinates. \underline{General Coordinate Invariance}. To achieve this
we need to define versions of vectors, covectors and tensors that work in curved spaces. Need to change the definition of the derivative
as well to make everything work.

%%%%%%%%%%%%%%%%%%%%%%%%-------  September 3, 2013 -----------%%%%%%%%%%%%%%%%%%%%
\section{September 3, 2013}
\subsection{Review of General Relativity (Cont.)}





\end{document}
