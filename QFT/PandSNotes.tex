\documentclass{article}
\usepackage{amssymb, amsmath}
\numberwithin{equation}{section}
\title{Notes on Peskin and Schroeder QFT}
\author{Rob Roy Fletcher}

%%%%%%%%----  Macros ----%%%%%%%%%%%%%%%
\newcommand{\Lagr}{\mathcal{L}}  %lagrangian
\newcommand{\dmu}{\partial_{\mu}} 
\newcommand{\Dmu}{\partial^{\mu}}
\newcommand{\beq}{\begin{equation}}
\newcommand{\eeq}{\end{equation}}

%%%%%%%---- Begin Document ----%%%%%%%%%%
\begin{document}

\section{Introduction}
\section{The Klein-Gordon Field}
\subsection{The Necessity of the Field Viewpoint}
This stuff is all pretty obvious. No notes for now.

\subsection{Elements of Classical Field Theory}
\subsubsection{Lagrangian Field Theory}
Derivation of the Euler-Lagrange equation for a single field. starting with equation (2.2) in the book
\begin{equation*}\label{ps2.2}\tag{PS 2.2}
	0=\delta S= \int d^4 x \lbrace \frac{\partial\Lagr}{\partial\phi} \delta\phi + \frac{\partial\Lagr}{\partial\left(\dmu\phi\right)} \delta\left(\dmu\phi\right) \rbrace
\end{equation*}
First we use Integration by parts on the last term.
\begin{equation}\label{Az1}
	\int d^4 x \frac{\partial\Lagr}{\partial\left(\dmu\phi\right)} \delta\left(\dmu\phi\right)
	= \frac{\partial\Lagr}{\partial(\dmu\phi)} \delta\phi - \int d^4 x \dmu\left(\frac{\partial\Lagr}{\partial\left(\dmu\phi\right)}\right) \delta\phi
\end{equation}
Then we can put the first term back in integral form by integrating and differentiating to make it look like it does in the book.
 Now to get the Euler-Lagrange equation, we notice that the first term in \eqref{Az1} is the surface term and should be evaluated
at the endpoints of $\delta S$. However, the endpoints of the action do not change, only the path between the endpoints changes so
$\delta S$ there is zero. Then we have,
\begin{equation}\label{Az2}
	0 = \int d^4 x \lbrace \frac{\partial\Lagr}{\partial\phi} \delta\phi - \dmu\left(\frac{\partial\Lagr}{\partial\left(\dmu\phi\right)}\right) \delta\phi \rbrace
	= \int d^4 x \lbrace \frac{\partial\Lagr}{\partial\phi} - \dmu\left(\frac{\partial\Lagr}{\partial\left(\dmu\phi\right)}\right) \rbrace \delta\phi
\end{equation}
Since we are choosing paths that have non-zero $\delta\phi$ the part of the integrand in brackets must be zero. This is the Euler-Lagrange equation.
\begin{equation}\label{ELeq}
	\frac{\partial\Lagr}{\partial\phi} - \dmu\left(\frac{\partial\Lagr}{\partial\left(\dmu\phi\right)}\right) = 0
\end{equation}


%%%%%%%%%%%----- Hamiltonian Field Theory ----%%%%%%%%%%%%%%%
\subsubsection{Hamiltonian Field Theory}

\subsubsection{Noether's Theorem}

%%%%%%%%%%%------ section 2.3 ----------%%%%%%%%%%%%%%%%%%%
\subsection{The Klein-Gordon Field as Harmonic Oscillators}

First transform the Klein-Gordon field from position space to Fourier or momentum space. Fourier transform is defined as:
\begin{equation}
    f(x) = \int \frac{d^4 k}{(2\pi)^4} e^{-ik \cdot x} \tilde{f}(x)
\end{equation}
and the inverse Fourier transform as:
\begin{equation}
    \tilde{f}(x) = \int d^4x e^{ik \cdot x} f(x)
\end{equation}
Now if we expand the Klein-Gordon Field as
\begin{equation}
    \phi(\mathbf{x},t) = \int \frac{d^3p}{(2\pi)^3} e^{i\mathbf{p} \cdot \mathbf{x}} \phi(\mathbf{p},t)
\end{equation}
Plugging this into the normal position space Klein-Gordon equation we have
\begin{equation}
    \left[ \frac{\partial^2}{\partial t^2} - \nabla ^2 + m^2 \right] 
    \int \frac{d^3p}{(2\pi)^3} e^{i\mathbf{p} \cdot \mathbf{x}} \phi(\mathbf{p},t) = 0
\end{equation}
If we distribute the integral over all three terms then look at the second one with the gradient in it we can see this will just pull down two factors of $ip$. The equation becomes
\begin{equation}
    \int \frac{d^3p}{(2\pi)^3} e^{i\mathbf{p} \cdot \mathbf{x}}  \left[ \frac{\partial^2}{\partial t^2} + |p|^2 + m^2 \right] \phi(\mathbf{p},t) = 0
\end{equation}
If we let 
\begin{equation}
    f(\mathbf{p},t) = \left[ \frac{\partial^2}{\partial t^2} + |p|^2 + m^2 \right] \phi(\mathbf{p},t)
\end{equation}
then this is just a Fourier transform of a function $f(\mathbf{x},t)$. Using the definition of the inverse Fourier Transform above we can see that if the Fourier transform of a function is zero then that function must be zero. Therefore we have $f(\mathbf{x},t) = 0$ or 
\begin{equation}
    \left[ \frac{\partial^2}{\partial t^2} + |p|^2 + m^2 \right] \phi(\mathbf{p},t) = 0.
\end{equation}


%%%%%%%%%%%%--------- Chapter 2 Solutions --------------%%%%%%%%%%%%%%%%%%%%%%%%%%

\subsection{Solutions to CH.2 Problems}

\textbf{2.1)} Fill this in later. Have done this problem in E+M and in GR.

\vspace{10mm}

\noindent
\textbf{ 2.2) The complex scalar field.} Consider the field theory of a complex-valued scalar field obeying the Klein-Gordon equation.
 The action of this theory is
\beq
    S = \int d^4x (\dmu \phi^* \Dmu \phi - m^2 \phi^* \phi).
\eeq
It is easiest to analyze this theory by considering $\phi(x)$ and $\phi^*(x)$, rather than the real and imaginary parts of $\phi(x)$ as the
basic dynamic variables.

\textbf{(a)} Find the conjugate momenta to $\phi(x)$ and $\phi^*(x)$ and the canonical commutation relations. Show that the Hamiltonian is
\beq
    H = \int d^3x (\pi^*\pi + \nabla\phi^* \dot \nabla \phi + m^2 \phi^*\phi).
\eeq
Compute the Heisenberg equation of motion for $\phi(x)$ and show that it is indeed the Klein-Gordon equation.
\vspace{5mm}
%%%%%%--solution---%%%%%%
    The conjugate momentum is defined as
    \beq
        \pi(x) \equiv \frac{\partial \Lagr}{\partial \dot \phi(x)}
    \eeq
    and our Lagrangian density in this case is
    \beq
        \Lagr = \dmu \phi^* \Dmu \phi - m^2 \phi^*\phi
    \eeq
    So then we have
    \beq
    \begin{split}
        \pi(x) = \frac{\partial}{\partial \dot \phi(x)}(\dmu \phi^* \Dmu \phi - m^2\phi^*\phi)
       \\ = \frac{\partial}{\partial \dot \phi(x)}( \partial_{t}\phi^* \partial^{t}\phi + \nabla\phi^* \cdot \nabla\phi- m^2\phi^*\phi)
       \\ =\frac{\partial}{\partial \dot \phi(x)}(\dot\phi^* \dot\phi+ \nabla\phi^* \cdot \nabla\phi- m^2\phi^*\phi)
       \\ = \dot\phi^*(x)
    \end{split}
    \eeq
    Using this same procedure we can see that we will also have $\pi^*(x) = \dot \phi(x)$.

    The cannonical commutators are,
    \beq
    \begin{split}
        [\phi(x), \pi(y)] = i \delta(x-y) \\
        [\phi^*(x), \pi^*(y)] = i \delta(x-y)\\
    \end{split}
    \eeq
    With all others equal to zero.















\end{document}

