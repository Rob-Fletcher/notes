\documentclass{article}
\usepackage{amssymb, amsmath}
\numberwithin{equation}{section}
\title{QFT Midterm}
\author{Rob Roy Fletcher}

%%%%%%%%----  Macros ----%%%%%%%%%%%%%%%
\newcommand{\Lagr}{\mathcal{L}}  %lagrangian
\newcommand{\dmu}{\partial_{\mu}} 
\newcommand{\dnu}{\partial_{\nu}}
\newcommand{\Dmu}{\partial^{\mu}}
\newcommand{\Dnu}{\partial^{\nu}} 
\newcommand{\dxa}{\partial_{x_a}} 
\newcommand{\dya}{\partial_{y_a}} 
\newcommand{\dxb}{\partial_{x_b}}
\newcommand{\dyb}{\partial_{y_b}}
\newcommand{\dz} {\partial_0}
\newcommand{\Dz} {\partial^0}

\newcommand{\ddl}[1]{\frac{\partial \Lagr}{\partial(\partial_{#1} \phi)}}
\newcommand{\cddl}[1]{\frac{\partial \Lagr}{\partial(\partial_{#1} \phi^*)}}
\newcommand{\forint}{\int \frac{d^3p}{(2\pi)^3} e^{i\vec{p} \cdot \vec{x}}}
\newcommand{\dpint}{\int \frac{d^3p}{(2\pi)^3}}

\newcommand{\ddf}{\delta^3(x-y)}

\newcommand{\beq}{\begin{equation}}
\newcommand{\eeq}{\end{equation}}

%%%%%%%---- Begin Document ----%%%%%%%%%%
\begin{document}
\maketitle
%%%%%%----- Classical ------%%%%%%%%%%%%%
\section{Classical Complex Scalar Field}
Using the canonical procedure, fully quantize the complex scalar field. I begin with the classical theory.
\subsection{Lagrangian}
A free scalar field has Lagrangian,
\beq \label{sl}
    \Lagr = (\partial_\mu \phi)^2 -m^2|\phi|^2 
\eeq
For a complex field this may be written explicitly in terms of $\phi$ and $\phi^*$ as
\beq \label{csl}
    \Lagr = \Dmu \phi^* \dmu \phi - m^2 \phi^* \phi
\eeq
We could also treat this field as two real fields $\phi(x) = \phi_1 + i \phi_2$, but I will not do that here.

%%%%%------ Eq of Motion -------%%%%%%%%%%
\subsection{Equation of Motion}
Now using the normal procedure we vary the action to get the equation of motion.
\beq \begin{split}
    &\delta S = \int d^4 x \left\{ \frac{\partial \Lagr}{\partial phi} \delta \phi + \ddl{\mu} \delta(\dmu \phi) \right\} \\
    &= \int d^4 x \left\{ \frac{\partial \Lagr}{\partial phi}\delta \phi-\dmu \left( \ddl{\mu}\right)\delta\phi+\dmu\left(\ddl{\mu} \delta\phi\right) \right\}=0
\end{split} \eeq
The last term here is a total derivative and so becomes a surface term. If we consider only deviations in which the endpoints are fixed then $\delta \phi$ is zero and the
third term is also zero. We are left with the Euler-Lagrange equation.
\beq
    \dmu\left( \ddl{\mu} \right) - \frac{\partial \Lagr}{\partial \phi} = 0
\eeq
Plugging in the Lagrangian in the form of (\ref{sl}) we get,
\beq \label{kg}
    \dmu\Dmu \phi + m^2\phi = (\dmu\Dmu + m^2)\phi = 0
\eeq 
which is just the Klein-Gordon equation as expected.

%%%%%%%----- Noether Charges ------%%%%%%%%
\subsection{Noether's Charges}
In class we derived an expression for the Noether current due to a infinitessimal space-time translation $\Delta \phi$ that leaves the lagrangian invariant,
\beq
    J^{0}_{\nu} = \ddl{0} \dnu \phi - \delta^{0}_{\nu}\Lagr
\eeq
Now because we are splitting the complex field into two independant fields we should rewrite this as,
\beq \label{noeth}
    J^{0}_{\nu} = \frac{\partial \Lagr}{\partial(\partial_{0}\phi_{\alpha})} \dnu \phi_{\alpha} - \delta^{0}_{\nu} \Lagr
\eeq
where we have indexed the fields by $\alpha$ and we can see we should just sum over the fields.
We also defined the conjugate momentum to $\phi$ as,
\beq
    \pi \equiv \ddl{0}
\eeq
Using our Lagrangian \ref{csl} for $\pi$ and $\pi^*$ we have,
\beq
    \pi \equiv \ddl{0} = \partial_{0} \phi^* \text{ ; } \pi^* \equiv \cddl{0} = \partial_{0} \phi
\eeq
Evaluating first the $\nu=0$ component of (\ref{noeth}),
\beq \begin{split}
    &J^{0}_{0} = \ddl{0} \partial_0 \phi + \cddl{0} \partial_0 \phi^* - \Lagr \\
    &= \pi \dz\phi + \pi^*\dz\phi^* - \Dmu\phi^* \dmu\phi + m^2 \phi^*\phi \\
    &= \pi\pi^* + \pi^*\pi - \Dz\phi^*\dz\phi- \partial^i\phi^* \partial_i\phi + m^2 \phi^*\phi \\
    &= \pi\pi^* + \pi^*\pi -\pi\pi^* - \nabla\phi^* \nabla \phi + m^2 \phi^* \phi \\
    &= \pi^*\pi - \nabla\phi^* \nabla \phi + m^2 \phi^* \phi = \mathcal{H}
\end{split} \eeq
which is the Hamiltonian density. The Hamiltonian is then,
\beq
    H = \int d^3x \left\{ \pi^*\pi - \nabla\phi^* \nabla \phi + m^2 \phi^* \phi   \right\}
\eeq
Now the $\nu=i$ components of the current which will be the momentum density,
\beq
    J^0_i = \ddl{0} \partial_i \phi + \cddl{0} \partial_i \phi^* = \pi \nabla\phi + \pi^*\nabla\phi^* = \mathcal{P}_i
\eeq
and similarly when integrated these currents make up the physical momentum.
\beq
    \vec{P} = \int d^3x (\pi\nabla\phi + \pi^*\nabla\phi^*)
\eeq
So now we have all of the Noether charges associated with space-time translations.
\beq
    Q_\mu = (H, \vec{P})
\eeq

For a complex scalar field there is another transformation that leaves the Lagrangian invariant. It is of the form $\phi(x) \rightarrow e^{i\alpha}\phi(x)$ with $\alpha$ a constant.
\beq
    \Lagr = \Dmu\phi^* \dmu\phi - m^2\phi^*\phi \rightarrow \Dmu(e^{-i\alpha}\phi^*) \dmu(e^{i\alpha}\phi) - m^2(e^{-i\alpha}\phi^*)(e^{i\alpha}\phi)
\eeq
It is easy to see that the factors of $e^{i\alpha}$ all cancel out and leave the lagrangian unchanged.
To calculate the current associated with this transformation we need to use the infinitesimal form,
\beq
    \delta \phi(x) = i\alpha\phi ; \delta \phi^* = -i\alpha\phi^*
\eeq
Now we can calculate the current using,
\beq \begin{split}
    &j^\mu = \ddl{\mu} \delta\phi + \cddl{\mu}\delta\phi^* = \ddl{\mu} i\alpha\phi - \cddl{\mu}i\alpha\phi^* \\
    &= i\alpha \left[ (\Dmu\phi^*)\phi - (\Dmu \phi)\phi^* \right] = i\alpha \left[ \phi^* \stackrel{\leftrightarrow}{\partial^{\mu}}\phi \right] \\
\end{split} \eeq 
From this we can see that the Noether's charge for this internal symmetry is,
\beq
    Q= \int d^3x j^0 = i\alpha \int d^3x \left[ \phi^* \stackrel{\leftrightarrow}{\partial^{0}}\phi \right]
\eeq
    
\subsection{Plane Wave Solutions}

To get some solutions to the Klein-Gordon equation we first want to Fourier transform $\phi(x)$.
\beq
    \phi(\vec{x},t) = \forint \phi(\vec{p},t)
\eeq
I will write $\phi(\vec{p},t)$ as $\phi_p$. If we substitute this transformation into the K-G equation we get.
\beq
    (\Dmu \dmu + m^2) \forint \phi_p = 0
\eeq
Splitting the derivatives up into time and space components and multiplying out the $(2\pi)^3$,
\beq \begin{split}
    &\left[\frac{\partial^2}{\partial t^2} - \frac{\partial^2}{\partial x^2} + m^2 \right] \int d^3p e^{ipx}\phi_p \\
    &= \int d^3p \left[\frac{\partial^2}{\partial t^2}e^{ipx}\phi_p - \frac{\partial^2}{\partial x^2}e^{ipx}\phi_p + m^2e^{ipx}\phi_p \right]\\
    &= \int d^3p e^{ipx} \left[ \frac{\partial^2}{\partial t^2}\phi_p + p^2\phi_p +m^2\phi_p \right] = 0\\
\end{split} \eeq
Since $e^{ipx}$ is never zero the rest of the integrand must be zero and so the Fourier transformed K-G equation is (and its complex conjugate),
\beq \begin{split}
   &\left( \frac{\partial^2}{\partial t^2} +p^2 +m^2 \right) \phi_p = 0 \\
   &\left( \frac{\partial^2}{\partial t^2} +p^2 +m^2 \right) \phi^*_p = 0 \\
\end{split} \eeq

These equations we recognize as the harmonic oscillator with frequency $\omega_p = \sqrt{p^2+m^2}$. We proceed the same way as in the normal quantum harmonic oscillator
by writing $\phi(x)$ as a linear combination of creation and annihilation operators. In this case however we are dealing with a continuum of plane waves with momentum $p$ so we will need an infinite number of these operators which will be indexed by $p$.
\beq
    \phi(t,\vec{x}) = \dpint \frac{1}{\sqrt{2\omega_p}} \left( a_p e^{ipx-i\omega_p t} + b^*_p e^{-ipx+i\omega_p t} \right)
\eeq
In the real scalar field case we got that the two Fourier coefficients were just complex conjugates of eachother, but now since we do not have the condition that $\phi(x)$
must be real this is no longer the case. Now evaluating $\pi(x)=\partial_0 \phi^*(x)$ we get,
\beq \begin{split}
    &\pi(t,\vec{x}) = \dpint \frac{1}{\sqrt{2\omega_p}} \left( i \omega_p a^*_p e^{-ipx+i\omega_p t} - i\omega_p b_p e^{ipx-i\omega_p t} \right)\\
     &= \dpint i \sqrt{\frac{\omega_p}{2}} \left( a^*_p e^{-ipx+i\omega_p t} - b_p e^{ipx-i\omega_p t} \right)\\
\end{split} \eeq
Now that we have $\phi(x)$ and $\pi(x)$ we can invert these to find expressions for $a_p$ and $b_p$. The easiest way to do this is to look at linear combinations of 
$\phi(x)$ and $\pi(x)$ and then take the inverse Fourier transform.
\beq \begin{split}
    &\sqrt{\frac{\omega_p}{2}}\phi(x)+\frac{i}{\sqrt{2\omega_p}}\pi^*(x) = \dpint \left\{ \frac{1}{2} \left( a_pe^{ipx}+b^*_p e^{-ipx}\right) +\frac{1}{2}\left(a_p e^{ipx} - b^*_p e^{-ipx} \right)  \right\} \\
    &= \dpint a_p e^{ipx} \\
\end{split} \eeq
Then inverse Fourier transform both sides to get
\beq
    a_p = \int d^3x \left( \sqrt{\frac{\omega_p}{2}} \phi(x) + \frac{i}{\sqrt{2\omega_p}} \pi^*(x) \right) e^{-ipx}
\eeq
Doing the same thing but with $\phi^*(x) + \pi(x)$ we get,
\beq
    b_p = \int d^3x\left( \sqrt{\frac{\omega_p}{2}} \phi^*(x) + \frac{i}{\sqrt{2\omega_p}} \pi(x) \right) e^{-ipx}
\eeq
\subsection{Time Dependence of $\hat{a}_p$}
To show that $\hat{a}_p$ does not depend on time we just have to take a time derivative and show that it is zero.
\beq \begin{split}
    & \partial_0 \hat{a}_p = \int d^3x \left( \sqrt{\frac{\omega_p}{2}} \partial_0 \phi+\frac{i}{\sqrt{2\omega_p}}\partial^2_0 \phi \right) e^{-ipx}+ \left(\sqrt{\frac{\omega_p}{2}} \phi+\frac{i}{\sqrt{2\omega_p}}\partial_0 \phi \right)(i\omega_p) e^{-ipx} \\
    &= \int d^3x\left(\sqrt{\frac{\omega_p}{2}} \partial_0 \phi+\frac{i}{\sqrt{2\omega_p}}\partial^2_0 \phi +\frac{i\omega^2}{\sqrt{2\omega_p}}\phi -\sqrt{\frac{\omega_p}{2}}\partial_0 \phi \right) e^{-ipx} \\
    &= \int d^3x\frac{i}{\sqrt{2\omega_p}} \left(\partial^2_0 \phi + \omega^2_p \phi \right) e^{-ipx}=\int d^3x\frac{i}{\sqrt{2\omega_p}} \left(\partial^2_0 \phi + (p^2+m^2) \phi \right) e^{-ipx} = 0 \\
\end{split} \eeq
where in the last step we have used the equation of motion.

%%%%%%%%%----------- Quantum Section -------------%%%%%%%%%%%%%%%%%%%
\section{Quantum Theory}
To go to the second quantization we must:
\begin{itemize}
    \item Promote $\phi(x)$ and $\pi(x)$ to operators.
    \item Demand that they satisfy the canonical commutation relations.
    \item In analogy with the quantum harmonic oscillator, find all operators and their commutators.
    \item Where needed apply normal ordering.
    \item Time order products of operators to preserve Lorentz invariance.
\end{itemize}
The canonical commutation relations are,
\beq \label{comm} \begin{split}
    &[\phi(x),\phi(y)] = [\pi(x),\pi(y)] = 0  \\
    &[\phi(x), \pi(y)] = i\delta^3(x-y)  \\
    &[\phi(x), \pi^*(y)] = 0 \text{, and any others by taking complex conjugates.}
\end{split} \eeq

%%%%%%%%%%%-------- Commutators ------------%%%%%%%%%%%%%%%%%%%%%%%%
\subsection{Other Commutators}

Another commutator we wish to evaluate is $[P_a, P_b]$. This is easy using (\ref{comm}).
\beq \begin{split}
    [P_a,P_b] = \int d^3x[\pi(x)\dxa\phi(x) - \pi^*(x)\dxb\phi^*(x), \pi(y)\dya\phi(y) - \pi^*(y)\dyb\phi^*(y)] \\
\end{split} \eeq
If we label each term in the commutator $A,B,C,\text{ and }D$ respectively we can examine this commutator in parts to make it easier. Multiplied out this will look like
$AC-AD-BC+BD-CA+DA-DB$. Lets first look at $AC-CA$.
\beq \begin{split}
    &AC-CA = \pi(x)\dxa\phi(x)\pi(y)\dyb\phi(y) - \pi(y)\dyb\phi(y)\pi(x)\dxa\phi(x) \\
    &= \pi(x)\pi(y) \left( \dxa\phi(x)\dyb\phi(y) - \dyb\phi(y)\dxa\phi(x) \right) \\
    & = \pi(x)\pi(y) \dxa\dyb[\phi(x),\phi(y)] = 0 \\
\end{split} \eeq
where I have used the commutators in (\ref{comm}) and the fact that partial derivatives commute. We can take the complex conjugate of the above calculation and this gives us that the terms $BD-DB=0$ as well. And finally from the last line in (\ref{comm}) we can see that the remaining terms which are cross terms between the fields and their 
complex conjugates will all trivially commute and are therefore also zero. Putting it all together we arrive at the conclusion that,
\beq
    [P_a,P_b] = 0
\eeq

Next we will look at the creation and annihilation operators. These commutators we can get from the canonical ones (\ref{comm}).
\newcommand{\ttop}[1]{\sqrt{\frac{\omega_{#1}}{2}}}
\newcommand{\bbot}[1]{\frac{i}{\sqrt{2\omega_{#1}}}}
\beq 
    [\hat{a}_p, \hat{a}^\dagger_q] = \int d^3x d^3y \left[\ttop{p}\phi(x) + \bbot{p} \pi^*(x), \ttop{q}\phi^*(y) - \bbot{q}\pi(y)\right]e^{-ipx}e^{iqy}
\eeq
Expanding this out, we can see that we can make commutators with each of the $\phi$s and $\pi$s. The only commutators that are non-zero will be $[\phi(x),\pi(y)]$ and $[\pi^*(x),\phi^*(y)]$. Writing thses out we get,

\beq \begin{split}
    &[\hat{a}_p,\hat{a}^\dagger_q] = \int d^3x d^3y \left(\frac{i}{2}\sqrt{\frac{\omega_p}{\omega_q}}[\phi(x),\pi(y)] + \frac{i}{2}\sqrt{\frac{\omega_q}{\omega_p}}[\pi^*(x),\phi^*(y)] \right)e^{-ipx+iqy} \\
    & = \int d^3x d^3y \left(\frac{i}{2}\sqrt{\frac{\omega_p}{\omega_q}}i\delta^3(x-y) + \frac{i}{2}\sqrt{\frac{\omega_q}{\omega_p}}i\delta^3(x-y) \right)e^{-ipx+iqy} \\
    &= \int d^3x \left(\frac{1}{2}\right)\left(\sqrt{\frac{\omega_p}{\omega_q}} +\sqrt{\frac{\omega_p}{\omega_q}} \right)e^{ix(q-p)} \\
    & = \delta^3(p-q) \\
\end{split}\eeq 
and the same calculation can be carried out for $[\hat{b}_p,\hat{b}^\dagger_q]=\delta^3(p-q)$. All other commutators are zero which can be seen again from (\ref{comm}) since the 
only commutators that will appear are the trivial ones. Just as in the real scalar field case these operators act on the $|0\rangle$ state as,
\beq
    \hat{a}_p|0\rangle = 0 \text{ ;         } \hat{b}_p|0\rangle = 0
\eeq    

We can define number operators as well for the complex scalar field. This field will have 2 number operators unlike the real field which only has one.
\beq
    N_a(p) \equiv \hat{a}^\dagger_p \hat{a}_p \text{      ;      }  
    N_b(p) \equiv \hat{b}^\dagger_p \hat{b}_p
\eeq
which will act on their eigenstates as
\beq
    N_a(p)|a(p),b(p)\rangle = a(p)|a(p),b(p)\rangle \text{ ; } N_b(p)|a(p),b(p)\rangle = b(p)|a(p),b(p)\rangle
\eeq
We have labeled these eigenstates as being common to both number operators which is easy to see is the case using the $\hat{a}$ and $\hat{b}$ commutators
\beq
    [N_a,N_b] = a^\dagger a b^\dagger b - b^\dagger b a^\dagger a = a^\dagger a b^\dagger b - a^\dagger a b^\dagger b = 0
\eeq
Finding the commutators for the number operators and the creation and annihilation operators will help us to find the spectrum of eigenvalues and eigenstates
\beq
    [N_a, a^\dagger] = a^\dagger a a^\dagger - a^\dagger a^\dagger a = a^\dagger(a^\dagger a + 1) - a^\dagger a^\dagger a = a^\dagger
\eeq
and similarly for the others,
\beq
    [N_a, a] = a \text{ ; } [N_b, b^\dagger] = b^\dagger \text{ ; } [N_b, b] = b
\eeq
Consider the state,
\beq \begin{split}
    &N_a \hat{a}^\dagger|a(p)\rangle = (\hat{a}^\dagger N_a + \hat{a}^\dagger)|a(p)\rangle = (a(p)+1)\hat{a}^\dagger|a(p)\rangle \\
    &N_a \hat{a}|a(p)\rangle = (\hat{a} N_a - \hat{a})|a(p)\rangle = (a(p)-1)\hat{a}|a(p)\rangle \\
\end{split} \eeq
similar for $N_b$. So these are similar ladder operators to the real scalar case but now we have two different particles created by the operators $\hat{a}^\dagger$
and $\hat{b}^\dagger$.
We write eigenstates of $\hat{a}^\dagger_p$ and $\hat{b}^\dagger_p$ as
\beq
    |n(p),m(p)\rangle = \Pi_p \hat{a}^{\dagger n(p)}_p \hat{b}^{\dagger m(p)}_p|0\rangle
\eeq

\subsection{Noether Charges Again}

Using our expressions for $\phi(x)$ and $\pi(x)$ we can calculate the Noether charges from the earlier section in terms of the number operators, creation and annihilation 
operators.
\beq \begin{split}
    &H = \int d^3x \left( \pi^*\pi -\nabla \phi^* \nabla \phi+m^2\phi^*\phi \right) \\
\end{split} \eeq
Going term by term,
\newcommand{\dg}{^\dagger}
\beq \begin{split}
    &\pi^*\pi = \dpint \frac{\omega}{2}(a a\dg -abff-b\dg a\dg f^*f^*+b\dg b)\\
    &\nabla \phi^* \nabla \phi = \dpint \frac{p^2}{2\omega_p}(a\dg a-a\dg b\dg f^*f^*-baff+bb\dg) \\
    &m^2\phi^* \phi = \dpint \frac{m^2}{2\omega_p}(a\dg a +a\dg b\dg f^*f^*+baff+bb\dg)
\end{split} \eeq
Using the fact that $\omega^2_p = \sqrt{p^2+m^2}$ allows us to combine and cancel terms. We are left with
\beq
    H = \dpint \omega_p(\hat{a}\dg \hat{a} + \hat{b} \hat{b}\dg) = \dpint \omega_p(N_a+N_b+1)
\eeq
The 1 comes from commuting $\hat{b}$ and $\hat{b}\dg$.This looks like just a sum of the number of each particle multiplied by its energy which makes sense.
Now we do the same thing for the charge $P$.
\beq
    P = \int d^3x(\pi\nabla\phi + \pi^*\nabla\phi^*) = \dpint p(a\dg a + b\dg b) = \dpint p(N_a+N_b)
\eeq
For the charge associated with the internal U(1) symmetry,
\beq
    Q = \int d^3x i(\phi^* \partial_0 \phi - \partial_0 \phi^* \phi) = \dpint (a\dg a - b\dg b) = \dpint (N_a - N_b)
\eeq

\subsection{Feynman Propagator}
\newcommand{\bracket}[3]{\langle #1 | #2 | #3 \rangle}
\newcommand{\ff}[2]{ f_{#1}(#2)}
We want to evaluate the amplitude for an a-type particle to be created at $x$ and destroyed at $y$, $\bracket{0}{\phi(t',y)\phi^*(t,x)}{0}$. For this to be Lorentz
invariant we must time order the product in the bracket.
\beq \label{top}
    \bracket{0}{T\phi(t',y)\phi^*(t,x)}{0} = \bracket{0}{\phi(t',y)\phi^*(t,x)}{0} \Theta(t'-t) + \bracket{0}{\phi^*(t,x)\phi(t',y)}{0} \Theta(t-t')
\eeq 
Evaluating the first term,
\beq \begin{split}
    &\bracket{0}{\phi(y),\phi^*(x)}{0} = \bracket{0}{\int dp dq \left[(a_p \ff{p}{y}+ b\dg_p \ff{p}{y}^*)(a\dg_q \ff{q}{x} + b_q \ff{q}{x})\right]   }{0} \\
    & = \bracket{0}{\int dp dq \left[ a_p \ff{p}{y} a\dg_q \ff{q}{x}^* + a_p\ff{p}{y}b_q\ff{q}{x} + b\dg_p \ff{p}{y}^* a\dg_q \ff{q}{x}^* + b\dg_p \ff{p}{y} b_q \ff{q}{x}  \right]  }{0} \\
\end{split}\eeq
Operating on the $|0\rangle$ with the all of the lowering operators we get every term except for the first vanishing.
\beq \begin{split}
    &= \bracket{0}{\int dp dq \ff{p}{y} \ff{q}{x}^* a_p a\dg_q  }  {0} = \bracket{0}{\int dp dq \ff{p}{y} \ff{q}{x}^* (a_p a\dg_q - a\dg_q a_p ) }  {0} \\
    &= \bracket{0}{\int dp dq  \ff{p}{y} \ff{q}{x}^* [a_p, a\dg_q]    }{0} = \bracket{0}{\int dp dq  \ff{p}{y} \ff{q}{x}^* \delta(p-q)    }{0} \\
    &= \int dp dq  \ff{p}{y} \ff{q}{x}^* \langle 0 | 0 \rangle = \int \frac{dp}{(2\pi)^3} \frac{1}{2\omega_p} e^{ip(y-x)} \\
\end{split} \eeq
Using this in the definition of the time ordered product (\ref{top}) we get,
\beq 
    \bracket{0}{T\phi(t',y)\phi^*(t,x)}{0} =\dpint\frac{1}{2\omega_p}\left[ \Theta(t'-t)e^{ip(y-x)} + \Theta(t-t')e^{-ip(x-y)}\right]
\eeq 
Inserting the integral definition of the $\Theta$ function
\beq
    \Theta(t-t') = \frac{i}{2\pi} \int d\omega \frac{e^{-i\omega(t-t')}}{\omega + i\epsilon}
\eeq
From here we go through the same process as in class where we rearrange to explicitly show which way the poles should be shifted in the complex integral.


\beq \begin{split}
    &\bracket{0}{T\phi(t',y)\phi^*(t,x)}{0} = \frac{i}{(2\pi)^4} \int \frac{dp d\omega}{2\omega_p} \left[ \frac{e^{ip(y-x)}e^{-i(\omega_p + \omega)(t'-t)} }{\omega + i\epsilon} + \frac{e^{ip(x-y)}e^{-i(\omega_p + \omega)(t-t')}}{-\omega + i\epsilon}     \right] \\ 
    & \text{Redefine } \omega = \omega_p + \omega \\
    & =  \frac{i}{(2\pi)^4} \int \frac{dp d\omega}{2\omega_p} e^{-ip(y-x)}\left[ \frac{e^{-i\omega(t'-t)} }{\omega - \omega_p + i\epsilon} - \frac{e^{-i\omega(t-t')}}{\omega + \omega_p - i\epsilon}     \right] \\
    &= \frac{i}{(2\pi)^4} \int d^4p e^{-ip(y-x)} \left[ \frac{1}{(\omega - [\omega_p -i\epsilon])(\omega - [-\omega_p +i\epsilon)}    \right] \\
    &=\frac{i}{(2\pi)^4} \int d^4p e^{-ip(y-x)} \left[  \frac{1}{\omega^2 - \omega_p + i(2\omega_p)\epsilon}\right] \text{, let } \epsilon = i(2\omega_p)\epsilon  \\
    &=\frac{i}{(2\pi)^4} \int d^4p e^{-ip(y-x)} \left[  \frac{1}{p^2-m^2 + i\epsilon}\right] \\
\end{split} \eeq  

Now with the amplitude solved for we define the Feynman Propagator,
\beq
    \Delta_F(y-x) = -i \bracket{0}{T\phi(t',y)\phi^*(t,x)}{0} =\int \frac{d^4p}{(2\pi)^4} e^{-ip(y-x)} \left[  \frac{1}{p^2-m^2 + i\epsilon}\right]
\eeq


\end{document}



